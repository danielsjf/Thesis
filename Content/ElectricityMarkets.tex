In Belgium, the electricity markets are governed by Elia. This is the Belgian Transmission System Operator or TSO. They own the high voltage network and have to organise the balance of the network.

In order to match the electricity supply and demand, some entity is needed to organise this. This organiser is called the Transmission System Operator (TSO). The TSO transports the electricity from the suppliers to the local electricity distributors.

To maintain supply and demand at all times (if not the frequency will change), a market system is needed. The Belgian electricity market consists of two parts. There is the Day-Ahead Market and the Intraday market.